\documentclass[letterpaper]{article}
\usepackage[top=1.0in,bottom=1.0in,left=1.0in,right=1.0in]{geometry}
\usepackage{verbatim}
\usepackage{amssymb}
\usepackage{graphicx}
\usepackage{svg}
\usepackage{longtable}
\usepackage{amsfonts}
\usepackage{amsmath}
\usepackage{hyperref}
\usepackage{float}
\usepackage{caption}
\usepackage{xcolor}
\def\thesection       {\arabic{section}}
\def\thesubsection     {\thesection.\alph{subsection}}

\author{Zoe Richter
        \\ \href{mailto:zrichte2@illinois.edu}{\texttt{zrichte2@illinois.edu}}
}
\title{Zoe's Advice:  Learn Some Programming and Scream Less\thanks{probably}}
\begin{document}
\maketitle
\section{What is This?}

Computer programming is a skill that can be applied in some capacity to most any field.  For nuclear applications, computers often help by either automating a process to limit human exposure to radiation, or by simulating complicated systems that are too costly or hazardous to be able to trivially build in a lab (think of trying to test things pertaining to reactors - they're big, really hot, highly pressurized, radioactive, and very expensive to make).\\

In the hands-on lesson, you worked with MatLab specifically.  Matlab is great for a lot of engineering applications, but you might not see it used outside of those applications (and it costs money, which makes me cry).\\

Below, I've collected some resources for teaching yourself programming, and some general tips that I've found helpful.

\section{What is an Unix?}






\end{document}